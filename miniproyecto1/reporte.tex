% Created 2021-09-22 mié 19:40
% Intended LaTeX compiler: pdflatex
\documentclass[11pt]{article}
\usepackage[utf8]{inputenc}
\usepackage[T1]{fontenc}
\usepackage{graphicx}
\usepackage{grffile}
\usepackage{longtable}
\usepackage{wrapfig}
\usepackage{rotating}
\usepackage[normalem]{ulem}
\usepackage{amsmath}
\usepackage{textcomp}
\usepackage{amssymb}
\usepackage{capt-of}
\usepackage{hyperref}
\author{Alejandra Campo Archbold, Juan Pablo Sierra Useche}
\date{\today}
\title{Solución al problema de Criptoaritmética}
\hypersetup{
 pdfauthor={Alejandra Campo Archbold, Juan Pablo Sierra Useche},
 pdftitle={Solución al problema de Criptoaritmética},
 pdfkeywords={},
 pdfsubject={},
 pdfcreator={Emacs 27.1 (Org mode 9.5)}, 
 pdflang={English}}
\begin{document}

\maketitle
\tableofcontents


\section{Introducción}
\label{sec:orgc3bba92}

La criptoaritmética es un problema que se deriva de la criptología; aunque a
diferencia de la criptología, que busca implementar sistemas seguros, la
criptoaritmética tiene un caracter más de entretenimiento.

Este ejercicio tiene tres componentes principales:

\begin{itemize}
\item Dos sumandos
\item Un resultado
\end{itemize}

Como cualquer suma, esta operación binaria toma dos elementos y al sumar sus
valores obtiene uno representativo como la suma de los dos. A diferencia de una
suma convencional, en la criptoaritmetica se inicia con las tres componentes  en
forma de letras tal que cada caracter \((A_i,B_j,C_k \in \left\{ A,B,C,\dots,X,Y,Z
\right\}\) para
todo \(i \in \left\{ 1,\dots,m \right\}\), \(j\in
\left\{1,\dots,n\right\}\) y  \(k\in \left\{1,\dots,o\right\}\)de la forma:

\begin{align*}
\begin{matrix}
&A_m&A_{m-1}&\dots&A_2&A_1\\
+&B_n&B_{n-1}&\dots&B_2&B_1\\
\hline
&C_o&C_{o-1}&\dots&C_2&C_1\\
\end{matrix}
\end{align*}

Las tres componentes no deben ser de igual tamaño, es decir (teniendo en cuenta
la notación de la ecuación anterior) \(m\), \(n\) y \(o\) no deben ser iguales;
pero deben cumplir tres reglas:

\begin{enumerate}
\item \(A_i + B_i = C_i\)
\item Si \(A_i + B_i \ne C_i\) existen 3 posibilidades:
\begin{itemize}
\item Si \(A_i + B_i >= 10\) entonces \(A_i + B_i = 10 + C_i\)
\item Si \(A_i + B_i < 10\) entonces \(A_{i-1} + B_{i-1} = 10 + C_{i-1}\)
\item Si \(A_i + B_i >= 10\) y \(A_{i-1} + B_{i-1} >= 10\) entonces \(A_i + B_i = 10 + C_i\) y \(A_{i-1} + B_{i-1} = 10 + C_{i-1}\).
\end{itemize}
\item Dos letras distintas no puedes ser asignadas a un mismo número.
\end{enumerate}

Ahora, teniendo estas reglas como base, la finalidad del problema es encontrar
una combinación donde todas las reglas se cumplan. Un ejemplo posible sería el siguiente:

\begin{align*}
\begin{matrix}
 &A_2&A_1\\
+&B_2&B_1\\ \hline
 &C_2&C_1\\
\end{matrix}
\to
\begin{matrix}
 &A&B\\
+&C&D\\ \hline
 &E&F\\
\end{matrix}
\to
\begin{matrix}
 &3&2\\
+&5&7\\ \hline
 &8&9\\
\end{matrix}
\end{align*}


\section{Métodos}
\label{sec:org4ef7443}

Con la finalidad de encontrar una forma para calcular las soluciones a este
problema, a conticuación, se van a explicar las dinamicas del ejercicio de foma
más rigurosa:

\subsection{Estado Inicial}
\label{sec:orga703106}

El estado inicial del problema consite del las tres componentes (ya descritas en
la introducción) como conjunto de caracteres entre la \(A\) y la \(Z\).

\subsection{Posibles Acciones}
\label{sec:org056505b}

Las posibles acciones se definen como una función del conjunto de letras
(aquellas que conforman a las tres componentes) a las que no se les ha asignado
un valor numérico apuntando al conjunto de números naturales en el intervalo
\(\left[0-9\right]\) que aún no han sido asignados a una letra.

\subsection{Función de Transición}
\label{sec:org71c8565}

Esta función retorna un estado del problema, donde se le a agregado la nueva
asignación letra\(\to\)número y a los conjuntos de letras y números disponibles
se les extrae los valores respectivos a la relación recien formada.

\subsection{Prueba de objetivo}
\label{sec:org6d32b3d}

Esta función determina si el estado actual cumple la relación entre los dos sumandos y el resultado.

\subsection{Función de costo}
\label{sec:org3649eee}

La función de costo está orientada a la implementación de una lista prioritaria,
donde se calcula el costo dependiendo de la frecuencia de cada una de las letras
que faltan por reemplazar, entre más frecuente menor es el costo (es decir más
prioritario segun lista prioritaria).

\subsection{Algoritmos de busqueda}
\label{sec:orgac3bfc5}

Vale la pena recalcar que el diseño del problema se implementó en formato nodal
(al igual que en clase), Por lo que el problema funciona con los algoritmos ya
programados desde la clase. Por lo tanto decidimos hacer pruebas utilizando los
siguientes algoritmos de busqueda:

\begin{itemize}
\item \textbf{\textbf{Depth First Search (DFS)}}: El DFS es un algoritmo que funciona en arboles, y funciona recorriendo de un extremo al otro del arbol, buscando siempre analizar el nodo más profundo.
\item \textbf{\textbf{Breadth First Search (BFS)}}: El algoritmo tambien hace un barrido de un lado al otro de un arbol, pero a diferencia del DFS, el BFS le da como privilegio a los niveles del arbol, por lo que revisa todos los nodos de cada nivel antes de bajar al siguiente.
\item \textbf{\textbf{Depth Limited Search (DLS)}}: Este algoritmo funciona de la misma forma que el \textbf{Depth First Search}, pero estableciendo un límite de profundidad.
\item \textbf{\textbf{Iterative Deepening Search (IDS)}}: Este algoritmo es la combinación de los dos anteriores, buscando repetidamente hasta cierta profundidad por todo el nivel.
\item \textbf{\textbf{Best First Search (BestFS)}}: Con este algoritmo se busca el nodo según la lista prioritaria lo determine.
\end{itemize}

\section{Resultados}
\label{sec:org621468d}

\begin{itemize}
\item \textbf{\textbf{Prueba 1}}: Resolver problemas con sumandos y resultados de igual tamaño (igual a dos) y seis letras diferentes.
\item \textbf{\textbf{Prueba 2}}: Resolver \texttt{SEND + MORE = MONEY}.
\end{itemize}

\begin{center}
\begin{tabular}{rll}
\textbf{\textbf{Prueba}} & \textbf{\textbf{Algoritmo}} & \textbf{\textbf{Resultados}}\\
\hline
1 & DFS & Promedio de \texttt{7 ms}.\\
1 & BFS & Nunca nos entregó resultados.\\
1 & DLS & Promedio de \texttt{9.80 ms}.\\
1 & IDS & Promedio de \texttt{23.7 s}.\\
1 & BestFS & Promedio de \texttt{1 min}.\\
\hline
2 & DFS & Promedio de \texttt{886 ms}.\\
2 & BFS & Nunca nos entregó resultados.\\
2 & DLS & Promedio de \texttt{298 ms}.\\
2 & IDS & Promedio de \texttt{11 mins}.\\
2 & BestFS & Nunca nos entregó resultados.\\
\hline
\end{tabular}
\end{center}

\section{Discución}
\label{sec:org7a3a307}

El algoritmo de \textbf{\textbf{BFS}} tuvo problemas de tiempo, y hay dos hipótesis:

\begin{enumerate}
\item Se cree que el problema está relacionado con la definición de prioridad, implicando que no encontramós la mejor de las formas para medir el problema.
\item Se cree que el problema se vincula con el mismo rendimiento de la función de costo, agregando demasiada complejidad al problema.
\end{enumerate}

Dejando abierta la mejor forma de solucionar el problema, pero entregando buenas soluciones con algoritmos más simples.

\section{Conclusiones}
\label{sec:org354be51}

Con problemas más simples, funciona mejor el DFS ya que el algoritmo es bastante
simple y es justo para lo que necesita el problema. Pero cuando la complejidad
del estado inicial va aumentando, el DLS resulta con mucho mejores promedios ya
que administra mejor la busqueda, asumiendo la cantidad de letras y así dejando
de intentar soluciones cuando estas se vuelven muy largas.
\end{document}
